% Options for packages loaded elsewhere
\PassOptionsToPackage{unicode}{hyperref}
\PassOptionsToPackage{hyphens}{url}
%
\documentclass[
  ignorenonframetext,
]{beamer}
\usepackage{pgfpages}
\setbeamertemplate{caption}[numbered]
\setbeamertemplate{caption label separator}{: }
\setbeamercolor{caption name}{fg=normal text.fg}
\beamertemplatenavigationsymbolsempty
% Prevent slide breaks in the middle of a paragraph
\widowpenalties 1 10000
\raggedbottom
\setbeamertemplate{part page}{
  \centering
  \begin{beamercolorbox}[sep=16pt,center]{part title}
    \usebeamerfont{part title}\insertpart\par
  \end{beamercolorbox}
}
\setbeamertemplate{section page}{
  \centering
  \begin{beamercolorbox}[sep=12pt,center]{part title}
    \usebeamerfont{section title}\insertsection\par
  \end{beamercolorbox}
}
\setbeamertemplate{subsection page}{
  \centering
  \begin{beamercolorbox}[sep=8pt,center]{part title}
    \usebeamerfont{subsection title}\insertsubsection\par
  \end{beamercolorbox}
}
\AtBeginPart{
  \frame{\partpage}
}
\AtBeginSection{
  \ifbibliography
  \else
    \frame{\sectionpage}
  \fi
}
\AtBeginSubsection{
  \frame{\subsectionpage}
}
\usepackage{amsmath,amssymb}
\usepackage{lmodern}
\usepackage{iftex}
\ifPDFTeX
  \usepackage[T1]{fontenc}
  \usepackage[utf8]{inputenc}
  \usepackage{textcomp} % provide euro and other symbols
\else % if luatex or xetex
  \usepackage{unicode-math}
  \defaultfontfeatures{Scale=MatchLowercase}
  \defaultfontfeatures[\rmfamily]{Ligatures=TeX,Scale=1}
\fi
% Use upquote if available, for straight quotes in verbatim environments
\IfFileExists{upquote.sty}{\usepackage{upquote}}{}
\IfFileExists{microtype.sty}{% use microtype if available
  \usepackage[]{microtype}
  \UseMicrotypeSet[protrusion]{basicmath} % disable protrusion for tt fonts
}{}
\makeatletter
\@ifundefined{KOMAClassName}{% if non-KOMA class
  \IfFileExists{parskip.sty}{%
    \usepackage{parskip}
  }{% else
    \setlength{\parindent}{0pt}
    \setlength{\parskip}{6pt plus 2pt minus 1pt}}
}{% if KOMA class
  \KOMAoptions{parskip=half}}
\makeatother
\usepackage{xcolor}
\newif\ifbibliography
\usepackage{longtable,booktabs,array}
\usepackage{calc} % for calculating minipage widths
\usepackage{caption}
% Make caption package work with longtable
\makeatletter
\def\fnum@table{\tablename~\thetable}
\makeatother
\setlength{\emergencystretch}{3em} % prevent overfull lines
\providecommand{\tightlist}{%
  \setlength{\itemsep}{0pt}\setlength{\parskip}{0pt}}
\setcounter{secnumdepth}{-\maxdimen} % remove section numbering
\ifLuaTeX
  \usepackage{selnolig}  % disable illegal ligatures
\fi
\IfFileExists{bookmark.sty}{\usepackage{bookmark}}{\usepackage{hyperref}}
\IfFileExists{xurl.sty}{\usepackage{xurl}}{} % add URL line breaks if available
\urlstyle{same} % disable monospaced font for URLs
\hypersetup{
  pdftitle={Resultados do Paebes - Rede Estadual},
  pdfauthor={Iago de Carvalho Nunes},
  hidelinks,
  pdfcreator={LaTeX via pandoc}}

\title{Resultados do Paebes - Rede Estadual}
\author{Iago de Carvalho Nunes}
\date{}

\begin{document}
\frame{\titlepage}

\begin{frame}
Introdução

Esta pesquisa é desenvolvida no projeto ``Estudos Educacionais'', por
meio da parceria entre Instituto Jones dos Santos Neves (IJSN),
Secretaria de Educação (Sedu) e Fundação de Amparo à Pesquisa (Fapes).

Objetivo: analisar a evolução do desempenho dos estudantes da rede
estadual nas provas do Paebes a fim de subsidiar os gestores da Sedu nas
análises para tomada de decisão e na proposição/ aperfeiçoamento/
reformulação de políticas públicas educacionais.

Objetivos específicos: 1. Observar as proficiências em Língua Portuguesa
e Matemática; 2. Acompanhar os efeitos da pandemia de COVID-19 no
ensino; 3. Visualizar as desigualdades de proficiência por
características dos estudantes. 4. Propor índice a partir do contextual.
\end{frame}

\begin{frame}
Índice

\begin{enumerate}[<+->]
\tightlist
\item
  Sobre os dados e as etapas de análise;
\item
  Evolução dos resultados por características dos estudantes;
\item
  Análise espacial dos resultados gerais de 2021, por municípios;
\item
  Sobre as características educacionais de 2021;
\item
  Resultados de 2021 por características educacionais e do estudante;
\item
  Sobre indicadores.
\end{enumerate}
\end{frame}

\hypertarget{sobre-os-dados-e-as-etapas-da-anuxe1lise-preliminar}{%
\section{Sobre os dados e as etapas da análise
preliminar}\label{sobre-os-dados-e-as-etapas-da-anuxe1lise-preliminar}}

\begin{frame}
Do banco de dados: filtro

Selecionou-se uma população-alvo para a análise preliminar:

\begin{enumerate}[<+->]
\tightlist
\item
  Estudantes da rede estadual de ensino;
\item
  Último ano do Ensino Fundamental (de 9 ou 8 anos) e da 3ª série do
  Ensino Médio Propedêutico;
\item
  Realizaram as provas de português ou matemática, possuem a
  proficiência calculada e responderam ao questionário contextual. A
  partir do código identificador único (chave) fornecido, une-se as duas
  bases (contextual e resultados) em uma.

  No comparativo entre a população de referência da avaliação e os
  estudantes efetivamente avaliados, o PAEBES demonstra-se um
  instrumento abrangente.
\end{enumerate}
\end{frame}

\begin{frame}{Tabela de Contingência}
\protect\hypertarget{tabela-de-continguxeancia}{}
\begin{longtable}[]{@{}
  >{\centering\arraybackslash}p{(\columnwidth - 6\tabcolsep) * \real{0.0875}}
  >{\centering\arraybackslash}p{(\columnwidth - 6\tabcolsep) * \real{0.4250}}
  >{\centering\arraybackslash}p{(\columnwidth - 6\tabcolsep) * \real{0.3750}}
  >{\centering\arraybackslash}p{(\columnwidth - 6\tabcolsep) * \real{0.1125}}@{}}
\caption{Matrículas, Rede Estadual, ES (Sinopse Estatística,
INEP).}\tabularnewline
\toprule()
\begin{minipage}[b]{\linewidth}\centering
Ano
\end{minipage} & \begin{minipage}[b]{\linewidth}\centering
Último ano do Ensino Fundamental
\end{minipage} & \begin{minipage}[b]{\linewidth}\centering
Terceiro ano do Ensino Médio
\end{minipage} & \begin{minipage}[b]{\linewidth}\centering
Total
\end{minipage} \\
\midrule()
\endfirsthead
\toprule()
\begin{minipage}[b]{\linewidth}\centering
Ano
\end{minipage} & \begin{minipage}[b]{\linewidth}\centering
Último ano do Ensino Fundamental
\end{minipage} & \begin{minipage}[b]{\linewidth}\centering
Terceiro ano do Ensino Médio
\end{minipage} & \begin{minipage}[b]{\linewidth}\centering
Total
\end{minipage} \\
\midrule()
\endhead
2015 & 16.889 & 25.675 & 42.564 \\
2016 & 16.189 & 25.101 & 41.290 \\
2017 & 15.482 & 25.835 & 41.317 \\
2018 & 16.600 & 24.931 & 41.531 \\
2019 & 17.721 & 22.794 & 40.515 \\
2021 & 19.935 & 27.737 & 47.672 \\
Total & 102.816 & 152.073 & 254.889 \\
\bottomrule()
\end{longtable}
\end{frame}

\begin{frame}{Tabela de Contingência}
\protect\hypertarget{tabela-de-continguxeancia-1}{}
Estudantes avaliados* no PAEBES

\begin{longtable}[]{@{}ccc@{}}
\toprule()
Ano & Último ano do Ensino Fundamental & Terceiro ano do Ensino Médio \\
\midrule()
\endhead
2015 & 14.743 & 20.367 \\
2016 & 14.185 & 19.980 \\
2017 & 13.269 & 21.931 \\
2018 & 14.819 & 21.152 \\
2019 & 15.506 & 19.144 \\
2021 & 16.840 & 20.451 \\
Total & 89.362 & 123.025 \\
\bottomrule()
\end{longtable}

*com a proficiência calculada e questionário preenchido.
\end{frame}

\hypertarget{evoluuxe7uxe3o-dos-resultados-por-caracteruxedsticas-dos-estudantes}{%
\section{Evolução dos resultados por características dos
estudantes}\label{evoluuxe7uxe3o-dos-resultados-por-caracteruxedsticas-dos-estudantes}}

\begin{frame}
Das características do estudante

Para uma melhor visualização da distribuição dos dados em gráficos de
caixa e violinos, como veremos, demanda-se categorias binárias de
análise. Neste sentido, as características dos discentes que se
classificam em mais de duas categorias foram transformadas em
binárias.Quatro características foram selecionadas para a análise. São
elas:

\begin{enumerate}[<+->]
\tightlist
\item
  A expressão de gênero do estudante, se masculino ou feminino;
\item
  A cor/raça do estudante, agregada em brancos e não brancos;
\item
  A distorção idade-série, se o estudante está com dois anos ou mais
  acima da idade-ideal para a série que cursa;
\item
  Se o domicílio do estudante é atendido pelo programa do Governo
  Federal de transferência de renda.
\end{enumerate}
\end{frame}

\begin{frame}{Tabela de Contingência}
\protect\hypertarget{tabela-de-continguxeancia-2}{}
Estudantes avaliados* no PAEBES - características selecionadas - Ensino
Fundamental

*com a proficiência calculada e questionário preenchido.
\end{frame}

\begin{frame}{Tabela de Contingência}
\protect\hypertarget{tabela-de-continguxeancia-3}{}
Estudantes avaliados* no PAEBES - características selecionadas - Ensino
Médio

*com a proficiência calculada e questionário preenchido.
\end{frame}

\begin{frame}
Da análise preliminar

O diagrama de caixa nos permite visualizar a variação da proficiência
dos estudantes entre as diferentes características selecionadas, onde
estão localizados 50\% dos valores, a mediana, a média e os valores
extremos.

O violino, que é formado pela estimativa de densidade em Kernel (EDK),
nos permite visualizar, mais atenuadamente, a probabilidade de os dados
assumir certo valor de proficiência, dadas certas características.

O foco da apresentação são os resultados para o Ensino Médio. Os do
Ensino Fundamental estão em apêndice.
\end{frame}

\hypertarget{resultados-por-guxeanero}{%
\section{Resultados por gênero}\label{resultados-por-guxeanero}}

\begin{frame}
Análise da distribuição: tendência central e dispersão.

Proficiência Média Estadual, Matemática, Ensino Médio: 285
\end{frame}

\begin{frame}
Análise da distribuição: tendência central e dispersão.

Proficiência Média Estadual, Língua Portuguesa, Ensino Médio: 286
\end{frame}

\begin{frame}
Análise da distribuição: tendência central e dispersão.

Disparidades:

Quando as meninas melhoram, os meninos melhoram ainda mais (mt), ou
quando elas retrocedem, os meninos ainda melhoram (lp).
\end{frame}

\hypertarget{resultados-por-rauxe7acor}{%
\section{Resultados por raça/cor}\label{resultados-por-rauxe7acor}}

\begin{frame}
Análise da distribuição: tendência central e dispersão.

Proficiência Média Estadual, Matemática, Ensino Médio: 285
\end{frame}

\begin{frame}
Análise da distribuição: tendência central e dispersão.

Proficiência Média Estadual, Língua Portuguesa, Ensino Médio: 286
\end{frame}

\begin{frame}
Análise da distribuição: tendência central e dispersão.

Disparidades:

Enquanto os brancos permanecem em trajetória ascendente (lp e mt), os
não brancos pioram no curto prazo (lp) e curto/longo prazo (mt).
\end{frame}

\hypertarget{resultados-por-distoruxe7uxe3o-idade-suxe9rie}{%
\section{Resultados por distorção
idade-série}\label{resultados-por-distoruxe7uxe3o-idade-suxe9rie}}

\begin{frame}
Análise da distribuição: tendência central e dispersão.

Proficiência Média Estadual, Matemática, Ensino Médio: 285
\end{frame}

\begin{frame}
Análise da distribuição: tendência central e dispersão.

Proficiência Média Estadual, Língua Portuguesa, Ensino Médio: 286
\end{frame}

\begin{frame}
Análise da distribuição: tendência central e dispersão.

Disparidades:

Desigualdades percebidas tanto em matemática quanto em português,
entretanto, os estudantes em distorção estão melhorando a pontuação
média mais rapidamente que os sem distorção.
\end{frame}

\hypertarget{resultados-por-estudantes-em-domicuxedlios-atendidos-por-programa-de-transferuxeancia-de-renda}{%
\section{\texorpdfstring{Resultados por estudantesem domicílios
atendidos porprograma de transferência de
renda}{Resultados por estudantes em domicílios atendidos por programa de transferência de renda}}\label{resultados-por-estudantes-em-domicuxedlios-atendidos-por-programa-de-transferuxeancia-de-renda}}

\begin{frame}
Análise da distribuição: tendência central e dispersão.

Proficiência Média Estadual, Matemática, Ensino Médio: 285
\end{frame}

\begin{frame}
Análise da distribuição: tendência central e dispersão.

Proficiência Média Estadual, Língua Portuguesa, Ensino Médio: 286
\end{frame}

\begin{frame}
Análise da distribuição: tendência central e dispersão.

Disparidades:

Ambos sentem o impacto da pandemia, mas os atendidos por programa de
transferência de renda sentem com maior intensidade.
\end{frame}

\hypertarget{resultados-por-municuxedpios}{%
\section{Resultados por municípios}\label{resultados-por-municuxedpios}}

\begin{frame}
Proficiência média por município - Ensino Médio, 2021

Proficiência Média Estadual, Matemática, Ensino Médio: 285
\end{frame}

\begin{frame}
Proficiência média por município - Ensino Médio, 2021

Proficiência Média Estadual, Língua Portuguesa, Ensino Médio: 286
\end{frame}

\hypertarget{anuxe1lise-de-caracteruxedsticas-educacionais-de-2021}{%
\section{Análise de características educacionais de
2021}\label{anuxe1lise-de-caracteruxedsticas-educacionais-de-2021}}

\begin{frame}
Das características educacionais do ano de 2021

Selecionou-se quatro perguntas do questionário relativas a
características do ensino para essa análise preliminar. São elas:

\begin{enumerate}[<+->]
\tightlist
\item
  Se, em 2021, as aulas foram presenciais, não presenciais ou por
  revezamento;
\item
  Por quantos meses o estudante ficou sem atividades escolares;
\item
  Quantas vezes o estudante realizou exercícios, durante o período sem
  aulas na escola;
\item
  Se os pais ou responsáveis arrumaram as condições necessárias para o
  discente estudar em casa.
\end{enumerate}
\end{frame}

\begin{frame}{Tabela de Contingência}
\protect\hypertarget{tabela-de-continguxeancia-4}{}
Estudantes avaliados* no PAEBES - características selecionadas - Ensino
Fundamental

*com a proficiência calculada e questionário preenchido.
\end{frame}

\begin{frame}{Tabela de Contingência}
\protect\hypertarget{tabela-de-continguxeancia-5}{}
Estudantes avaliados* no PAEBES - características selecionadas - Ensino
Médio

*com a proficiência calculada e questionário preenchido.
\end{frame}

\hypertarget{resultados-por-guxeanero-1}{%
\section{Resultados por gênero}\label{resultados-por-guxeanero-1}}

\begin{frame}
Análise da distribuição: tendência central e dispersão.

Proficiência Média Estadual, Matemática, Ensino Médio: 285
\end{frame}

\begin{frame}
Análise da distribuição: tendência central e dispersão.

Proficiência Média Estadual, Matemática, Ensino Médio: 285
\end{frame}

\begin{frame}
Análise da distribuição: tendência central e dispersão.

Proficiência Média Estadual, Matemática, Ensino Médio: 285
\end{frame}

\begin{frame}
Análise da distribuição: tendência central e dispersão.

Proficiência Média Estadual, Matemática, Ensino Médio: 285
\end{frame}

\hypertarget{resultados-por-rauxe7acor-1}{%
\section{Resultados por raça/cor}\label{resultados-por-rauxe7acor-1}}

\begin{frame}
Análise da distribuição: tendência central e dispersão.

Proficiência Média Estadual, Matemática, Ensino Médio: 285
\end{frame}

\begin{frame}
Análise da distribuição: tendência central e dispersão.

Proficiência Média Estadual, Matemática, Ensino Médio: 285
\end{frame}

\begin{frame}
Análise da distribuição: tendência central e dispersão.

Proficiência Média Estadual, Matemática, Ensino Médio: 285
\end{frame}

\begin{frame}
Análise da distribuição: tendência central e dispersão.

Proficiência Média Estadual, Matemática, Ensino Médio: 285
\end{frame}

\hypertarget{resultados-por-distoruxe7uxe3o-idade-suxe9rie-1}{%
\section{Resultados por distorção
idade-série}\label{resultados-por-distoruxe7uxe3o-idade-suxe9rie-1}}

\begin{frame}
Análise da distribuição: tendência central e dispersão.

Proficiência Média Estadual, Matemática, Ensino Médio: 285
\end{frame}

\begin{frame}
Análise da distribuição: tendência central e dispersão.

Proficiência Média Estadual, Matemática, Ensino Médio: 285
\end{frame}

\begin{frame}
Análise da distribuição: tendência central e dispersão.

Proficiência Média Estadual, Matemática, Ensino Médio: 285
\end{frame}

\begin{frame}
Análise da distribuição: tendência central e dispersão.

Proficiência Média Estadual, Matemática, Ensino Médio: 285
\end{frame}

\hypertarget{resultados-por-estudantes-em-domicuxedlios-atendidos-por-programa-de-transferuxeancia-de-renda-1}{%
\section{\texorpdfstring{Resultados por estudantesem domicílios
atendidos porprograma de transferência de
renda}{Resultados por estudantes em domicílios atendidos por programa de transferência de renda}}\label{resultados-por-estudantes-em-domicuxedlios-atendidos-por-programa-de-transferuxeancia-de-renda-1}}

\begin{frame}
Análise da distribuição: tendência central e dispersão.

Proficiência Média Estadual, Matemática, Ensino Médio: 285
\end{frame}

\begin{frame}
Análise da distribuição: tendência central e dispersão.

Proficiência Média Estadual, Matemática, Ensino Médio: 285
\end{frame}

\begin{frame}
Análise da distribuição: tendência central e dispersão.

Proficiência Média Estadual, Matemática, Ensino Médio: 285
\end{frame}

\begin{frame}
Análise da distribuição: tendência central e dispersão.

Proficiência Média Estadual, Matemática, Ensino Médio: 285
\end{frame}

\hypertarget{sobre-indicadores}{%
\section{Sobre indicadores}\label{sobre-indicadores}}

\begin{frame}
Sugestão

\begin{enumerate}
\tightlist
\item
  Um painel de indicadores para todas as variáveis é extenso;
\item
  A síntese das informações em um índice auxilia na análise dos dados,
  inclusive para a distribuição conjunta de fatores.
\end{enumerate}
\end{frame}

\begin{frame}
Indicador de Nível Socioeconômico - INSE/Inep

\begin{itemize}
\tightlist
\item
  Componentes do INSE: três dimensões, 15 indicadores (continua):

  \begin{itemize}
  \tightlist
  \item
    Qual é a maior escolaridade de sua/seu\ldots{}

    \begin{itemize}
    \tightlist
    \item
      Q008 mãe (ou mulher responsável por você)?
    \item
      Q009 pai (ou homem responsável por você)?
    \end{itemize}
  \item
    Dos itens relacionados abaixo, quantos existem na sua casa?

    \begin{itemize}
    \tightlist
    \item
      Q019 -- Geladeira
    \item
      Q021 -- Computador (ou notebook)
    \item
      Q022 -- Quartos para dormir
    \item
      Q023 -- Televisão
    \item
      Q024 -- Banheiro
    \item
      Q025 -- Carro
    \end{itemize}
  \end{itemize}
\end{itemize}
\end{frame}

\begin{frame}
Indicador de Nível Socioeconômico - INSE/Inep

\begin{itemize}
\tightlist
\item
  Componentes do INSE: três dimensões, 15 indicadores (continuação):

  \begin{itemize}
  \tightlist
  \item
    Na sua casa tem:

    \begin{itemize}
    \tightlist
    \item
      Q027 -- Rede wi-fi?
    \item
      Q029 -- Mesa para estudar (ou escrivaninha)?
    \item
      Q030 -- Garagem?
    \item
      Q031 -- Forno de micro-ondas?
    \item
      Q032 -- Aspirador de pó?
    \item
      Q033 -- Máquina de lavar roupa?
    \item
      Q034 -- Freezer (independente ou segunda porta da geladeira)?
    \end{itemize}
  \end{itemize}
\end{itemize}
\end{frame}

\begin{frame}[fragile]
Indicador de Nível Socioeconômico - INSE/Inep

\begin{itemize}[<+->]
\item
  Etapas da montagem:
\item
\begin{verbatim}
   Estudo dos respondentes ao questionário e das médias;
\end{verbatim}
\item
\begin{verbatim}
   Aplicação de pesos;
\end{verbatim}
\item
\begin{verbatim}
   Aplicação da Teoria de Resposta ao Item;
\end{verbatim}
\item
\begin{verbatim}
   Definição de níveis e faixas do indicador;
\end{verbatim}
\item
\begin{verbatim}
   Classificação dos municípios.
\end{verbatim}
\end{itemize}
\end{frame}

\begin{frame}
A metodologia pode ser reproduzida no Paebes?

\end{frame}

\begin{frame}
Possibilidades

Análise de tipos de indicadores e agregações:

\begin{enumerate}[<+->]
\tightlist
\item
  Indicadores educacionais puros, socioeducacionais, entre outros tipos;
\item
  Médias geométricas, aritiméticas e ponderadas;
\item
  Distribuições conjuntas/marginais;
\item
  Ajustes por desigualdades;
\item
  Padronização de dimensões.
\end{enumerate}
\end{frame}

\hypertarget{apuxeandices}{%
\section{Apêndices}\label{apuxeandices}}

\hypertarget{resultados-por-guxeanero-ensino-fundamental-anos-finais}{%
\section{\texorpdfstring{Resultados por gêneroEnsino Fundamental, Anos
Finais}{Resultados por gênero  Ensino Fundamental, Anos Finais}}\label{resultados-por-guxeanero-ensino-fundamental-anos-finais}}

\begin{frame}
Análise da distribuição: tendência central e dispersão.

Proficiência Média Estadual, Matemática, Ensino Fundamental: 257
\end{frame}

\begin{frame}
Análise da distribuição: tendência central e dispersão.

Proficiência Média Estadual, Língua Portuguesa, Ensino Fundamental: 255
\end{frame}

\hypertarget{resultados-por-rauxe7acor-ensino-fundamental-anos-finais}{%
\section{\texorpdfstring{Resultados por raça/corEnsino Fundamental, Anos
Finais}{Resultados por raça/cor  Ensino Fundamental, Anos Finais}}\label{resultados-por-rauxe7acor-ensino-fundamental-anos-finais}}

\begin{frame}
Análise da distribuição: tendência central e dispersão.

Proficiência Média Estadual, Matemática, Ensino Fundamental: 257
\end{frame}

\begin{frame}
Análise da distribuição: tendência central e dispersão.

Proficiência Média Estadual, Língua Portuguesa, Ensino Fundamental: 255
\end{frame}

\hypertarget{resultados-por-distoruxe7uxe3o-idade-suxe9rie-ensino-fundamental-anos-finais}{%
\section{\texorpdfstring{Resultados por distorção idade-sérieEnsino
Fundamental, Anos
Finais}{Resultados por distorção idade-série  Ensino Fundamental, Anos Finais}}\label{resultados-por-distoruxe7uxe3o-idade-suxe9rie-ensino-fundamental-anos-finais}}

\begin{frame}
Análise da distribuição: tendência central e dispersão.

Proficiência Média Estadual, Matemática, Ensino Fundamental: 257
\end{frame}

\begin{frame}
Análise da distribuição: tendência central e dispersão.

Proficiência Média Estadual, Língua Portuguesa, Ensino Fundamental: 255
\end{frame}

\hypertarget{resultados-por-estudantes-em-domicuxedlios-atendidos-por-programa-de-transferuxeancia-de-renda-ensino-fundamental-anos-finais}{%
\section{\texorpdfstring{Resultados por estudantesem domicílios
atendidos porprograma de transferência de rendaEnsino Fundamental, Anos
Finais}{Resultados por estudantes em domicílios atendidos por programa de transferência de renda  Ensino Fundamental, Anos Finais}}\label{resultados-por-estudantes-em-domicuxedlios-atendidos-por-programa-de-transferuxeancia-de-renda-ensino-fundamental-anos-finais}}

\begin{frame}
Análise da distribuição: tendência central e dispersão.

Proficiência Média Estadual, Matemática, Ensino Fundamental: 257
\end{frame}

\begin{frame}
Análise da distribuição: tendência central e dispersão.

Proficiência Média Estadual, Língua Portuguesa, Ensino Fundamental: 255
\end{frame}

\hypertarget{resultados-por-municuxedpios-ensino-fundamental-anos-finais}{%
\section{\texorpdfstring{Resultados por municípiosEnsino Fundamental,
Anos
Finais}{Resultados por municípios  Ensino Fundamental, Anos Finais}}\label{resultados-por-municuxedpios-ensino-fundamental-anos-finais}}

\begin{frame}
Proficiência média por município - Matemática, Ensino Fundamental, 2021

Proficiência Média Estadual, Matemática, Ensino Fundamental: 257
\end{frame}

\begin{frame}
Proficiência média por município - Língua Portuguesa, Ensino
Fundamental, 2021

Proficiência Média Estadual, Língua Portuguesa, Ensino Fundamental: 255
\end{frame}

\end{document}
